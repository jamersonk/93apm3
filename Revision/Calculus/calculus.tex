\documentclass[14pt]{article}
\usepackage{graphicx} % Required for inserting images
\usepackage{algpseudocode}
\usepackage{amssymb}
\usepackage{amsfonts}
\usepackage{amsmath}
\usepackage{amsthm}
\usepackage{geometry}
\usepackage{color}
\usepackage{setspace}
\onehalfspacing
\usepackage{titlesec}
\usepackage{paralist}
\usepackage{mathtools}
\usepackage{hyperref}
\usepackage{multicol}
\usepackage{multirow}
\titleformat{\section}[block]{\color{black}\Large\bfseries\filcenter}{}{1em}{}

\geometry{
    a4paper,
    nomarginpar,
    includeheadfoot,
    total = {140mm, 257mm},
    headheight = 1.2cm
}

\hypersetup{
    colorlinks=true,
    linkcolor=blue,
    filecolor=magenta,      
    urlcolor=cyan,
    pdfpagemode=FullScreen,
    }

    \theoremstyle{definition}
    \newtheorem*{remark}{Remarks}
    \newtheorem*{example}{Example}
    \newtheorem*{discussion}{Discussion}
    \newtheorem{definition}{Definition}[subsection]
    \newtheorem{proposition}[definition]{Proposition}
    \newtheorem{theorem}[definition]{Theorem}
    \newtheorem{notation}[definition]{Notation}
    \newtheorem{coro}[definition]{Corollary}
    \newtheorem{lemma}[definition]{Lemma}
    \newtheorem{axiom}[definition]{Axiom}
    \newtheorem*{exercise}{Exercise}


\newcommand{\impl}{\rightarrow}
\newcommand{\eq}{\thicksim}
\newcommand{\quotient}{A/\thicksim}
\newcommand{\fun}[3]{#1\colon #2\rightarrow#3}
\newcommand{\bb}[1]{\mathbb{#1}}
\newcommand{\restrict}{\upharpoonright}
\newcommand{\xor}{\oplus}

\title{Calculus}
\author{based on the work of George B. Thomas Jr.}
\date{AY25/26} % delete this line to display the current date

%%% BEGIN DOCUMENT
\begin{document}

\maketitle

This set of notes is adapted from \textit{Thomas' Calculus} by George B. Thomas Jr.
\newpage

\tableofcontents

\newpage


%%% SETS
\section{Functions}
\subsection{Functions \& Graphs}
\begin{definition}
A \textbf{function} $f$ form a domain set $D$ to a range set $Y$ is a rule that assigns a unique value $f(x)$ of $D$ for each element in $D$. 
\end{definition}
\begin{discussion}
The domain $D$ is the set of possible input values. The range $Y$ is the set of possible output values. Each element in $D$ has a \textbf{single}, \textbf{unique} value $f(x)$.The graph of $f$ with domain $D$ is the set 
\[
\{(x,f(x)|x\in D\}.
\]
\end{discussion}
\begin{axiom}
A function $f$ can only have one value $f(x)$ for each $x$ in its domain. Any vertical line $x=a$ can only intersect $f$ once.
\end{axiom}
\begin{definition}
A function $f$ is a \textbf{piecewise function} if it has separate definitions for different parts of its domain.
\[ f(x)=\begin{cases}
f_1 & x\leq0\\
f_2 & x > 1
\end{cases}
\]
\end{definition}
\begin{definition}
A function $y=f(x)$ is an
\begin{align*}
\text{even function if }	&f(-x)=f(x),\\
\text{odd function if }	&f(-x)=-f(x),
\end{align*}
for every $x$ in the function's domain.
\end{definition}
\begin{definition}
Two variables $x$, $y$ are \textbf{proportional} if they are always a constant multiple of one another, i.e. $y=kx$. If $1>k>-1$, $x$ and $y$ are said to be \textbf{inversely proportional}. 
\end{definition}

\subsection{Composite Functions}
\begin{definition}
The \textbf{composite function} is the function $f(g(x))$.
\end{definition}
\begin{remark}
$x$ is the input for the function $g(x)$; $g(x)$ is then used as the input to $f$.
\end{remark}
\begin{definition}
A graph is shifted vertically when we add $k$ to $f$.
\[
y=f(x)+k
\]
\end{definition}
\begin{definition}
A graph is shifted horizontally by $h$ when we add $h$ to $x$.
\[
y=f(x+h)
\]
\end{definition}


\subsection{Trigonometric Functions}
\begin{definition}
The radian is defined as
\begin{equation}
s=r\theta,
\end{equation}
where $\theta$ is in radians.
\end{definition}
\begin{definition}
The degree is defined as
\begin{equation*}
1^{\circ}=\frac{\pi}{180}rad.
\end{equation*}
\end{definition}
\begin{definition}
By convention, the positive angle is measured counter-clockwise from the positive $x$-axis.
\end{definition}
\begin{definition}
The basic trigonometric functions are
\begin{align*}
\sin{\theta}&=\frac{opp}{hyp}	&\csc{\theta}&=\frac{hyp}{opp}=\frac{1}{\sin{\theta}} \\
\cos{\theta}&=\frac{adj}{hyp}	&\sec{\theta}&=\frac{hyp}{adj}=\frac{1}{\cos{\theta}}\\
\tan{\theta}&=\frac{opp}{adj}=\frac{\sin{\theta}}{\cos{\theta}} &\cot{\theta}&=\frac{adj}{opp}=\frac{1}{\tan{\theta}}=\frac{\cos{\theta}}{\sin{\theta}}
\end{align*}
\end{definition}

\subsection{Trigonometric Identities}
\begin{definition}
The \textbf{Pythagorean identity} is
\begin{equation}\label{eqn:trig_pyth}
\sin^2{\theta}+\cos^2{\theta}=1
\end{equation}
\end{definition}
\begin{remark}
Dividing Eqs. (\ref{eqn:trig_pyth}) by $cos^2{\theta}$ and $\sin^2{\theta}$ gives
\begin{align*}
1+\tan^2\theta&=sec^2\theta\\
1+\cot^2\theta&=\csc^2\theta
\end{align*}
\end{remark}
\begin{definition}
The \textbf{addition formulas} hold for al angles $A$ and $B$.
\begin{equation}\label{eqn:trig_add}
\begin{split}
\cos{\left(A\pm B\right)}&=\cos{A}\cos{B}\mp\sin{A}\sin{B}\\
\sin{\left(A\pm B\right)}&=\sin{A}\cos{B}\pm\cos{A}\sin{B}
\end{split}
\end{equation}
\end{definition}
\begin{remark}
All other (basic) trigonometric identities derive from Eqs. (\ref{eqn:trig_pyth}) and Eqs. (\ref{eqn:trig_add}).
\end{remark}
\begin{definition}
The \textbf{double-angle formulas} derive from substituting $\theta$ for $A$ and $B$ in Eqs (\ref{eqn:trig_add}) and give
\begin{equation}
\begin{split}
\cos{2\theta}=\cos^2{\theta}-\sin^2{\theta}\\
\sin{2\theta}=2\sin{\theta}\cos{\theta}.
\end{split}
\end{equation}
\end{definition}
\begin{definition}
The \textbf{half-angle formulas} derive by combining
\begin{equation*}
\begin{split}
\cos^2{\theta}+\sin^2{\theta}&=1\\
\cos^2{\theta}-\sin^2{\theta}&=cos{2\theta}
\end{split}
\end{equation*}
which give
\begin{align}
\cos^2\theta&=\frac{1+\cos{2\theta}}{2}\rightarrow \cos{\frac{\theta}{2}}	=\pm\sqrt{\frac{1+\cos{2\theta}}{2}} \\
\sin^2\theta&=\frac{1-\cos{2\theta}}{2}\rightarrow \sin{\frac{\theta}{2}}=\pm\sqrt{\frac{1-\cos{2\theta}}{2}}
\end{align}
\end{definition}
\begin{definition}
The \textbf{Law of Cosines} (Cosine Law) is
\begin{equation}
c^2=a^2+b^2-2ab\cos{\theta}.
\end{equation}
\end{definition}


\subsection{Exponential Functions}
\begin{definition}
The exponential function of base $a$ is
\[
f(x)=a^x,a>0.
\]
\end{definition}
Where $f(x)=a^x$, the following rules apply:
\begin{enumerate}
	\item $a^x\times a^y=a^{x+y}$
	\item $\frac{a^x}{a^y}=a^{x-y}$
	\item ${\left(a^x\right)}^y={\left(a^y\right)}^x=a^{xy} $
	\item $a^x\times b^x=\left(ab\right)^x$
	\item $\frac{a^x}{b^x}=\left(\frac{a}{b}\right)^x$
\end{enumerate}
\begin{definition}
The \textbf{euler's number} $e$ is defined as
\begin{equation*}
\begin{split}
e	&=\lim_{x\to\infty}\left(\frac{n}{n+1}\right)^n\\
	&= 1 + \frac{1}{1!}+\frac{1}{2!}+...+\frac{1}{n!}+...\\
\end{split}
\end{equation*}
\end{definition}
\begin{definition}
The \textbf{natural exponential function} refers to the function
\[
f(x)=e^x
\]
\end{definition}
\begin{definition}
Given the function $y=y_0e^{kx}$,
\begin{center}
Exponential growth is when $k>0$, and\\
Exponential decay is when $k<0$.
\end{center}
\end{definition}

\subsection{Inverse Functions \& Logarithms}


\end{document}